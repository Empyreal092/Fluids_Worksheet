\documentclass[11pt,letterpaper]{article}
\usepackage[utf8]{inputenc}
\usepackage[left=1in,right=1in,top=1in,bottom=1in]{geometry}
\usepackage{amsfonts,amsmath}
\usepackage{graphicx,float}
\usepackage{esint}
% -----------------------------------
\usepackage{hyperref}
\hypersetup{%
  colorlinks=true,
  linkcolor=blue,
  citecolor=blue,
  urlcolor=blue,
  linkbordercolor={0 0 1}
}
% -----------------------------------
\usepackage[authordate,backend=biber]{biblatex-chicago}
\addbibresource{citation.bib}
% -----------------------------------
\usepackage{fancyhdr}
\newcommand\course{MATH-UA.0230, PHYS-UA 180\\Introduction to Fluid Dynamics}
\newcommand\hwnumber{1}                  % <-- homework number
\newcommand\NetIDa{Ryan Sh\`iji\'e D\`u} 
\newcommand\NetIDb{February 2nd, 2023}
\pagestyle{fancyplain}
\headheight 35pt
\lhead{\NetIDa\\\NetIDb}
\chead{\textbf{\Large Worksheet \hwnumber}}
\rhead{\course}
\lfoot{}
\cfoot{}
\rfoot{\small\thepage}
\headsep 1.5em
% -----------------------------------
\usepackage{titlesec}
\renewcommand\thesubsection{(\arabic{section}.\alph{subsection})}
\titleformat{\subsection}[runin]
        {\normalfont\bfseries}
        {\thesubsection}% the label and number
        {0.5em}% space between label/number and subsection title
        {}% formatting commands applied just to subsection title
        []% punctuation or other commands following subsection title
% -----------------------------------
\setlength{\parindent}{0.0in}
\setlength{\parskip}{0.1in}
% -----------------------------------
\input{../command.tex}
\begin{document}

\section{Rate of strain tensor: examples}
\subsection{Shear flow}
Take a velocity field with $u=y$ and $v,w=0$. 
\begin{enumerate}
    \item Calculate its velocity gradient tensor.
    \item What is its symmetric and anti-symmetric parts?
    \item What is its divergence and vorticity vector? 
\end{enumerate}

\subsection{Straining flow}
Take a velocity field with $u=x$, $v=-y$, and $w=0$. 
\begin{enumerate}
    \item Do the same calculation.
    \item {[Adapted from \cite{Aris_62}, Exercise 4.42.1]} Take a line connecting the origin and a fluid parcel, show that if $\theta$ is the angle between the line and the $x$-axis, then the rate of change of $\log\tan\theta$ is constant along a particle path. 
\end{enumerate}

\subsection{}
[From \cite{Aris_62}, Exercise 4.45.1] Take the velocity field
\begin{align}
    \ve v = \alpha\ve \omega + \ve \omega \times \ve x.
\end{align}
\begin{enumerate}
    \item Do the same calculation.
    \item Interpret the motion of a fluid parcel in this flow.
\end{enumerate}

\subsection{Rankine vortex}
Now we use the cylindrical coordinate with $(r,\theta,z)$. We have the flow field
\begin{align}
    u_\theta = \begin{cases}
        \Omega r, & r<a\\
        \dsp{\frac{\Omega a^2}{r}}, & r\geq a
    \end{cases}
\end{align}
and $u_r = u_z = 0$. 
\begin{enumerate}
    \item Do the same calculation (by converting the velocity into the Cartesian coordinate).
\end{enumerate}
We will come back to this example to practice polar coordinate and to study vorticity and circulation.  

\section{From divergence theorem to Green's theorem}
\subsection{}
[From \cite{Aris_62}, Exercise 3.31.3] By taking $\ve a$ to be independent of $x$, and $a_3 = 0$, show using divergence theorem that if $A$ is an area in the $(x,y)$-plane bounded by a curve $C$, then
\begin{align}
    \iint_A \frac{\pe a_1}{\pe x}+\frac{\pe a_2}{\pe y}\;\de A = \oint_C a_1t_2-a_2t_1\;\de s
\end{align}
where $\ve t=(t_1,t_2)$ is the unit tangent vector to $C$. 

\subsection{}
[From \cite{Aris_62}, Exercise 3.31.4] Deduce from the previous question that
\begin{align}
    \iint_A \frac{\pe a_2}{\pe x}-\frac{\pe a_1}{\pe y}\;\de A = \oint_C a_1t_1+a_2t_2\;\de s
\end{align}
The right hand side can be written as
\begin{align}
    \oint_C a_1\de x+a_2\de y
\end{align}
and we have derived the Green's theorem.

\section{Green's identity}
\subsection{}
We take a 3D scalar field $\psi$ and a 3D vector field $\ve\Gamma$ with sufficient smoothness defined on some region $U \subset \mathbb{R}^3$. Show the identity
\begin{align}
    \iiint_U \left( \psi \, \nabla \cdot \ve{\Gamma} + \ve{\Gamma} \cdot \nabla \psi\right)\, dV  = \oiint_{\partial U} \psi \left( \ve{\Gamma} \cdot \ve{n} \right)\, dS=\oiint_{\partial U}\psi\ve{\Gamma}\cdot d\ve{S}.\label{eq:green_1st_gen}
\end{align}

\subsection{}
Use \eqref{eq:green_1st_gen} to show the Green's first identity. Take 3D scalar fields $\psi$ and $\varphi$ both with sufficient smoothness:
\begin{align}
    \iiint_U \left( \psi \, \nabla^2 \varphi + \nabla \psi \cdot \nabla \varphi \right)\, dV  = \oiint_{\partial U} \psi \left( \nabla \varphi \cdot \ve{n} \right)\, dS=\oiint_{\partial U}\psi\,\nabla\varphi\cdot d\ve{S}.
\end{align}

\subsection{}
Show the Green's second identity:
\begin{align}
    \iiint_U \left( \psi \, \nabla^2 \varphi - \varphi \, \nabla^2 \psi\right)\, dV = \oiint_{\partial U} \left( \psi \nabla \varphi - \varphi \nabla \psi\right)\cdot d\ve{S}.
\end{align}
This shows that the Laplacian is a self-adjoint operator for functions vanishing on the boundary so that the right hand side of the above identity is zero.


\section{Expanding gas}
\subsection{In a tube}
We have a velocity field $u=x$ and $v,w=0$. 
\begin{enumerate}
    \item Write down the ``differential form'' of the mass conservation equation.
    \item Use the method of characteristic to solve for the density $\rho(x,t)$.
\end{enumerate}

\subsection{In free 3D space}
Now have a velocity field $u=x$, $v=y$, and $w=z$. Do the same calculation.


    
\vfill
\printbibliography


\end{document}